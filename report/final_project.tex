\documentclass{article}
\usepackage{amsmath}
\usepackage{amsfonts}
\usepackage{graphicx}
\usepackage{url}
\usepackage{mcode}
\usepackage{mathtools}
\usepackage{amsmath}
\usepackage{color}
\usepackage{xfrac}
\usepackage{flushend}
\usepackage{wrapfig}
\usepackage{minted}
\usepackage{mdframed}
\usepackage{etoolbox,xpatch}
\usepackage{hyperref}
\hypersetup{
    colorlinks,
    citecolor=black,
    filecolor=black,
    linkcolor=black,
    urlcolor=black
  }


\setminted{fontsize=\small,baselinestretch=1}

\lstset{language=C,keywordstyle={\bfseries \color{black}}}

\NewDocumentCommand{\codeword}{v}{
  \texttt{\textcolor{black}{#1}}
}

\makeatletter
\AtBeginEnvironment{minted}{\dontdofcolorbox}
\def\dontdofcolorbox{\renewcommand\fcolorbox[4][]{##4}}
\xpatchcmd{\inputminted}{\minted@fvset}{\minted@fvset\dontdofcolorbox}{}{}
\makeatother

\lstset{
   basicstyle=\fontsize{11}{13}\selectfont\firacode
}

\begin{document}


\title{Final Project: Security Keypad Lock}
\author{John McAvoy \\
  Email: mcavoyj5@students.rowan.edu}

\maketitle
\tableofcontents

\section{Project Description}
This project was destined to be a security keypad, similar to the keypads found
on digital safes and doors. The MSP430G2553 launchpad was used as well as a
custom 12-button keypad that is used to enter and set combinations. The security
keypad is able to set combinations, lock and unlock an electric lock, and
communicate through UART.

\section{Final Code}

\subsection{src/msp430g2553\_keypad\_lock.c}

\subsubsection{main}

\begin{listing}[H]
  \caption{src/msp430g2553\_keypad\_lock.c::main()}
  \inputminted[linenos,firstline=1, lastline=29]{C}{../msp430g2553/src/msp430g2553_keypad_lock.c}
\end{listing}

The main function starts by stopping the watch dog timer. Then, it calls the
setup functions for the lock pins, keypad pins, timeout timer, and UART. Next,
it sends a welcome message to UART and then enables the global interrupt.

\subsubsection{Timer0\_A1}

\begin{listing}[H]
  \caption{src/msp430g2553\_keypad\_lock.c::Timer0\_A1()}
  \inputminted[linenos,firstline=29, lastline=42]{C}{../msp430g2553/src/msp430g2553_keypad_lock.c}
\end{listing}

The Timer0\_A1 interrupt service routine is called every time the timeout timer
finishes its cycle. The service routine stops the timeout timer and then locks
triggers the lock state.

\subsubsection{USCI0RX\_ISR}

\begin{listing}[H]
  \caption{src/msp430g2553\_keypad\_lock.c::USCI0RX\_ISR()}
  \inputminted[linenos,firstline=43, lastline=54]{C}{../msp430g2553/src/msp430g2553_keypad_lock.c}
\end{listing}

The USCI0RX interrupt service routine was not used, however this project could
use this in the future to send commands via UART to the device.

\subsubsection{Port\_1}

\begin{listing}[H]
  \caption{src/msp430g2553\_keypad\_lock.c::Port\_1()}
  \inputminted[linenos,firstline=55, lastline=69]{C}{../msp430g2553/src/msp430g2553_keypad_lock.c}
\end{listing}

The Port\_1 interrupt service routine is called every time one of keypad keys
that is connected to a P1 port is pressed. The service routine calls
\codeword{handle_key_press}.

\subsubsection{Port\_2}

\begin{listing}[H]
  \caption{src/msp430g2553\_keypad\_lock.c::Port\_2()}
  \inputminted[linenos,firstline=70, lastline=]{C}{../msp430g2553/src/msp430g2553_keypad_lock.c}
\end{listing}

The Port\_2 interrupt service routine calls the same functions as the Port\_1
interrupt service routine. This service will run whenever the keypad keys
connected to P2 are pressed.

\subsection{lib/uart.c}

\begin{listing}[H]
  \caption{lib/uart.h}
  \inputminted[linenos,firstline=, lastline=]{C}{../msp430g2553/lib/uart.h}
\end{listing}

\subsubsection{setup\_uart}

\begin{listing}[H]
  \caption{lib/uart.c::setup\_uart()}
  \inputminted[linenos,firstline=, lastline=18]{C}{../msp430g2553/lib/uart.c}
\end{listing}

Setup UART configures the UART to use P1.1 as RXD and P1.2 as TXD, uses the 1MHx
clock and sets the baud rate at 9600.


\subsubsection{send\_bytes}

\begin{listing}[H]
  \caption{lib/uart.c::send\_bytes()}
  \inputminted[linenos,firstline=19, lastline=]{C}{../msp430g2553/lib/uart.c}
\end{listing}

Send bytes takes a pointer to a byte array and loops through the length of the
array. For each character in the array, the loop waits for the TX buffer to be
ready, then it writes the byte to \codeword{UCA0TXBUF} which sends the byte via
UART.

\subsection{lib/security.c}

\begin{listing}[H]
  \caption{lib/security.h}
  \inputminted[linenos,firstline=, lastline=,fontsize=\tiny,
  baselinestretch=1]{C}{../msp430g2553/lib/security.h}
\end{listing}

\subsubsection{setup\_lock\_pins}

\begin{listing}[H]
  \caption{lib/security.c::setup\_lock\_pins()}
  \inputminted[linenos,firstline=, lastline=37]{C}{../msp430g2553/lib/security.c}
\end{listing}

The lock, unlock, and timeout pins are set to I/O mode and set to output
direction. Then, the lock state is set.

\subsubsection{setup\_timeout\_timer}

\begin{listing}[H]
  \caption{lib/security.c::setup\_timeout\_timer()}
  \inputminted[linenos,firstline=38, lastline=47]{C}{../msp430g2553/lib/security.c}
\end{listing}

The timeout timer is used to delay combination inputs. The timer is configured
to use the 32kHz clock to count to 3200 with the capture compare interrupt
enabled. This makes the timeout delay 1 second.

\subsubsection{set\_passcode}

\begin{listing}[H]
  \caption{lib/security.c::set\_passcode()}
  \inputminted[linenos,firstline=48, lastline=57]{C}{../msp430g2553/lib/security.c}
\end{listing}

\codeword{set_passcode} is used to read a \codeword{Key} array, loop through the
array and set the passcode to the \codeword{new_passcode}.

\subsubsection{lock}

\begin{listing}[H]
  \caption{lib/security.c::lock()}
  \inputminted[linenos,firstline=58, lastline=66]{C}{../msp430g2553/lib/security.c}
\end{listing}

The lock state sets the unlocked pin off and the locked pin on.

\subsubsection{unlock}

\begin{listing}[H]
  \caption{lib/security.c::unlock()}
  \inputminted[linenos,firstline=67, lastline=75]{C}{../msp430g2553/lib/security.c}
\end{listing}

The unlock state sets the unlocked pin on and the locked pin off.

\subsubsection{start\_timeout\_timer}

\begin{listing}[H]
  \caption{lib/security.c::start\_timeout\_timer()}
  \inputminted[linenos,firstline=76, lastline=84]{C}{../msp430g2553/lib/security.c}
\end{listing}

The timeout timer is turned on by seting \codeword{TA0CTL} into continuous mode.
When the timer is started, the timout pin is set high.

\subsubsection{stop\_timeout\_timer}

\begin{listing}[H]
  \caption{lib/security.c::stop\_timeout\_timer()}
  \inputminted[linenos,firstline=85, lastline=93]{C}{../msp430g2553/lib/security.c}
\end{listing}

The timeout timer is turned off by setting \codeword{TACTL} to halt and
\codeword{TAR}.

\subsubsection{handle\_combination}

\begin{listing}[H]
  \caption{lib/security.c::handle\_combination()}
  \inputminted[linenos,firstline=95, lastline=]{C}{../msp430g2553/lib/security.c}
\end{listing}

\subsection{lib/keypad.c}

\begin{listing}[H]
  \caption{lib/keypad.h}
  \inputminted[linenos,firstline=8, lastline=,fontsize=\scriptsize,
  baselinestretch=1]{C}{../msp430g2553/lib/keypad.h}
\end{listing}

\subsubsection{setup\_keypad\_pins}

\begin{listing}[H]
  \caption{lib/keypad.c::setup\_keypad\_pins()}
  \inputminted[linenos,firstline=, lastline=40]{C}{../msp430g2553/lib/keypad.c}
\end{listing}

The keypad pins are connected to I/O pins in Port\_1 and Port\_2. All of the
pins are set to I/O mode and the direction is set to input. Also, the Port\_1
and Port\_2 interrupts are also enabled for each of the keypad keys.

\subsubsection{clear\_key\_interrupt\_flags}

\begin{listing}[H]
  \caption{lib/keypad.c::setup\_keypad\_pins()}
  \inputminted[linenos,firstline=41, lastline=49]{C}{../msp430g2553/lib/keypad.c}
\end{listing}

The keypad key interrupts are cleared by setting the corresponding bits
\codeword{P1IFG} and \codeword{P2IIFG} low.

\subsubsection{get\_key\_pressed}

\begin{listing}[H]
  \caption{lib/keypad.c::get\_key\_pressed}
  \inputminted[linenos,firstline=50, lastline=69]{C}{../msp430g2553/lib/keypad.c}
\end{listing}

The \codeword{get_key_pressed} function returns the corresponding \codeword{Key}
that is pressed by using bit masks on \codeword{P1} and \codeword{P2}.

\subsubsection{handle\_key\_press}

\begin{listing}[H]
  \caption{lib/keypad.c::handle\_key\_press()}
  \inputminted[linenos,firstline=70, lastline=84]{C}{../msp430g2553/lib/keypad.c}
\end{listing}

The \codeword{handle_key_press} function gets the \codeword{Key} that is pressed
and adds it to the \codeword{combination_in} array. When the
\codeword{combination_in} is full, then the \codeword{combination_in} is passed
to \codeword{handle_combination}.

\subsubsection{send\_combination\_in}

\begin{listing}[H]
  \caption{lib/keypad.c::send\_combination\_in()}
  \inputminted[linenos,firstline=85, lastline=]{C}{../msp430g2553/lib/keypad.c}
\end{listing}

The \codeword{send_combinaiton_in} function creates an output character buffer
that is sends the current combination entered via UART.

\subsection{lib/Key.c}

\begin{listing}[H]
  \caption{lib/Key.h}
  \inputminted[linenos,firstline=, lastline=]{C}{../msp430g2553/lib/Key.h}
\end{listing}

\subsubsection{key2Char}

\begin{listing}[H]
  \caption{lib/Key.c::key2Char}
  \inputminted[linenos,firstline=, lastline=]{C}{../msp430g2553/lib/Key.c}
\end{listing}

The \codeword{Key} enum holds all the different types of key categories which
makes it useful for passing keycodes between functions instead of numbers.
\codeword{key2Char} is used to convert a \codeword{Key} into a printable
character.

\section{Unit Tests}

\subsection{tests/keypad/keypad\_test.c}
\begin{listing}[H]
  \caption{tests/keypad/keypad\_test.c}
  \inputminted[linenos,firstline=,
  lastline=, fontsize=\scriptsize]{C}{../msp430g2553/tests/keypad/keypad_test.c}
\end{listing}

This unit test was used to test the \codeword{keypad.c} functions. Each time a
key was pressed, its value was sent via UART for debugging.

\subsection{tests/security/security\_test.c}
\begin{listing}[H]
  \caption{tests/keypad/security/security\_test.c}
  \inputminted[linenos,firstline=,
  lastline=,
  fontsize=\tiny]{C}{../msp430g2553/tests/security/security_test.c}
\end{listing}

This unit test was used to test the \codeword{security.c} functions. Various
\codeword{Key} combinations were sent to the \codeword{handle_combination}
function to test that they work properly.

\subsection{tests/timer/timer\_test.c}
\begin{listing}[H]
  \caption{tests/timer/timer\_test.c}
  \inputminted[linenos,firstline=,
  lastline=,
  fontsize=\footnotesize]{C}{../msp430g2553/tests/timer/timer_test.c}
\end{listing}

This unit test was used to make sure that the timeout timer was configured
correctly. Each time the interrupt was triggered, a message was via to UART.

\subsection{tests/uart/uart\_test.c}
\begin{listing}[H]
  \caption{tests/uart/uart\_test.c}
  \inputminted[linenos,firstline=,
  lastline=,
  fontsize=]{C}{../msp430g2553/tests/uart/uart_test.c}
\end{listing}

This unit test was used to test the \codeword{uart.c} functions.

\section{Conclusions and Future Work}

The keypad security lock works correctly, however it can be improved by adding
encoding logic to reduce the number of I/O pins used. The MSP430G2553 does not
have enough I/O pins for all 12 keys in the number pads. I originally designed
the device to use a primary encoder that would reduce the 12 signals from the
keypad into a 4-bit signal. I wasn't able to accomplish this circuit because all
I had was 2-input OR gates and the circuit was too complicated to fit on a
breadboard.

\end{document}
